% Day 2: Advanced LaTeX Techniques - COMPLETE FIXED VERSION
% SUZA Workshop - Scientific Writing with LaTeX & AI
% All UTF-8 errors fixed, all lstlisting properly closed

\documentclass[aspectratio=169]{beamer}

% Theme and Color Settings
\usetheme{Madrid}
\usecolortheme{default}
\setbeamercolor{structure}{fg=blue!70!black}
\setbeamercolor{frametitle}{bg=blue!10,fg=blue!70!black}
\usefonttheme{serif}

% Packages
\usepackage[utf8]{inputenc}
\usepackage{graphicx}
\usepackage{listings}
\usepackage{xcolor}
\usepackage{hyperref}
\usepackage{amsmath,amssymb}
\usepackage{booktabs}

% Code Listing Settings - FIXED
\lstset{
	basicstyle=\ttfamily\small,
	keywordstyle=\color{blue},
	commentstyle=\color{gray},
	stringstyle=\color{red},
	frame=single,
	breaklines=true,
	numbers=left,
	numberstyle=\tiny\color{gray},
	escapechar=@,
	xleftmargin=0pt,
	framexleftmargin=0pt
}

% Header and Footer
\setbeamertemplate{headline}{
	\leavevmode%
	\hbox{%
		\begin{beamercolorbox}[wd=.02\paperwidth,ht=2.5ex,dp=1.125ex]{section in head/foot}%
		\end{beamercolorbox}%
		\begin{beamercolorbox}[wd=.15\paperwidth,ht=2.5ex,dp=1.125ex,left]{section in head/foot}%
			\includegraphics[height=2ex]{../resources/suza_logo.png}
		\end{beamercolorbox}%
		\begin{beamercolorbox}[wd=.83\paperwidth,ht=2.5ex,dp=1.125ex,center]{section in head/foot}%
			\scriptsize State University of Zanzibar (SUZA) | Scientific Writing with LaTeX \& AI
		\end{beamercolorbox}%
	}
	\vskip2pt
	{\color{blue!70!black}\hrule height 0.5pt}
}

\setbeamertemplate{footline}{
	{\color{blue!70!black}\hrule height 0.5pt}
	\vskip2pt
	\leavevmode%
	\hbox{%
		\begin{beamercolorbox}[wd=.5\paperwidth,ht=2.5ex,dp=1.125ex,left,leftskip=1em]{author in head/foot}%
			\scriptsize Masoud Hamad  |  massoud.hamad@suza.ac.tz  |  Department of CS\&IT, SCCMS
		\end{beamercolorbox}%
		\begin{beamercolorbox}[wd=.4\paperwidth,ht=2.5ex,dp=1.125ex,right]{date in head/foot}%
			\scriptsize\insertshortdate
		\end{beamercolorbox}%
		\begin{beamercolorbox}[wd=.1\paperwidth,ht=2.5ex,dp=1.125ex,right,rightskip=1em]{page number in head/foot}%
			\insertframenumber{} / \inserttotalframenumber
		\end{beamercolorbox}
	}%
	\vskip2pt
}

\title{Day 2: Advanced LaTeX Techniques}
\subtitle{Bibliography Management, Reports, Theses \& Presentations}
\author{Masoud Hamad}
\institute{Department of Computer Science \& Information Technology\\
	School of Computing, Communication and Media Studies\\
	State University of Zanzibar}
\date{\today}

\begin{document}
	
	\begin{frame}
		\titlepage
	\end{frame}
	
	\begin{frame}{Today's Agenda}
		\tableofcontents
	\end{frame}
	
	%==============================================================================
	\section{Recap: Day 1 Essentials}
	
	\begin{frame}{Quick Review}
		\textbf{Yesterday we covered:}
		\begin{itemize}
			\item Document structure and basic commands
			\item Text formatting and sections
			\item Mathematical equations
			\item Tables and figures
			\item Cross-references
		\end{itemize}
		
		\vspace{1em}
		
		\textbf{Today we advance to:}
		\begin{itemize}
			\item Professional bibliography management
			\item Multi-file document organization
			\item Academic reports and theses
			\item Creating presentations with Beamer
			\item Custom commands and packages
		\end{itemize}
	\end{frame}
	
	%==============================================================================
	\section{Bibliography with BibTeX}
	
	\begin{frame}{Why BibTeX?}
		\begin{block}{The Problem}
			Managing citations manually is time-consuming and error-prone.
		\end{block}
		
		\vspace{1em}
		
		\textbf{BibTeX Benefits:}
		\begin{itemize}
			\item Centralized reference database
			\item Automatic formatting of citations
			\item Easy switching between styles
			\item Reuse references across documents
			\item Integration with reference managers
		\end{itemize}
	\end{frame}
	
	\begin{frame}[fragile]{Creating a .bib File}
		Create \texttt{references.bib}:
		
		\begin{lstlisting}
			@article{einstein1905,
				author  = {Albert Einstein},
				title   = {On Electrodynamics},
				journal = {Annalen der Physik},
				year    = {1905},
				volume  = {17},
				pages   = {891--921}
			}
			
			@book{knuth1984,
				author    = {Donald E. Knuth},
				title     = {The TeXbook},
				publisher = {Addison-Wesley},
				year      = {1984}
			}
		\end{lstlisting}
	\end{frame}
	
	\begin{frame}{BibTeX Entry Types}
		\textbf{Common entry types:}
		
		\begin{columns}
			\begin{column}{0.48\textwidth}
				\begin{itemize}
					\item \texttt{@article} - Journal paper
					\item \texttt{@book} - Book
					\item \texttt{@inproceedings} - Conference
					\item \texttt{@phdthesis} - PhD thesis
					\item \texttt{@mastersthesis} - Master's
				\end{itemize}
			\end{column}
			
			\begin{column}{0.48\textwidth}
				\begin{itemize}
					\item \texttt{@techreport} - Report
					\item \texttt{@misc} - Other
					\item \texttt{@online} - Web resource
					\item \texttt{@incollection} - Chapter
					\item \texttt{@manual} - Manual
				\end{itemize}
			\end{column}
		\end{columns}
	\end{frame}
	
	\begin{frame}[fragile]{Using BibTeX in Documents}
		\begin{lstlisting}
			@\textbackslash@documentclass{article}
			@\textbackslash@usepackage{cite}
			
			@\textbackslash@begin{document}
			
			Recent research @\textbackslash@cite{smith2023} shows...
			Multiple studies @\textbackslash@cite{jones2022,brown2021}...
			
			@\textbackslash@bibliographystyle{plain}
			@\textbackslash@bibliography{references}
			
			@\textbackslash@end{document}
		\end{lstlisting}
	\end{frame}
	
	\begin{frame}[fragile]{Compilation Process}
		\textbf{To compile with BibTeX:}
		
		\begin{lstlisting}[language=bash]
			pdflatex mydocument.tex
			bibtex mydocument
			pdflatex mydocument.tex
			pdflatex mydocument.tex
		\end{lstlisting}
		
		\vspace{1em}
		
		\begin{block}{In Overleaf}
			Overleaf handles this automatically! Just click "Recompile."
		\end{block}
	\end{frame}
	
	\begin{frame}{Citation Styles}
		Change style with \texttt{\textbackslash bibliographystyle\{...\}}:
		
		\begin{description}
			\item[plain] Alphabetical, numbered
			\item[alpha] Alpha-numeric labels [Ein05]
			\item[ieeetr] IEEE Transactions style
			\item[apalike] APA-like (author-year)
			\item[abbrv] Abbreviated names
			\item[unsrt] Unsorted (citation order)
		\end{description}
	\end{frame}
	
	\begin{frame}[fragile]{natbib Package}
		For flexible citations:
		
		\begin{lstlisting}
			@\textbackslash@usepackage{natbib}
			
			% Textual citation
			@\textbackslash@citet{smith2023} showed...
			% Output: Smith (2023) showed...
			
			% Parenthetical citation
			Recent studies @\textbackslash@citep{jones2022}...
			% Output: Recent studies (Jones, 2022)...
			
			@\textbackslash@bibliographystyle{plainnat}
			@\textbackslash@bibliography{references}
		\end{lstlisting}
	\end{frame}
	
	%==============================================================================
	\section{Document Organization}
	
	\begin{frame}[fragile]{Multi-File Projects}
		For large documents, split into files:
		
		\textbf{Main file (main.tex):}
		\begin{lstlisting}
			@\textbackslash@documentclass{report}
			@\textbackslash@usepackage{...}
			
			@\textbackslash@begin{document}
			
			@\textbackslash@input{chapters/chapter1}
			@\textbackslash@input{chapters/chapter2}
			@\textbackslash@input{chapters/chapter3}
			
			@\textbackslash@bibliography{references}
			
			@\textbackslash@end{document}
		\end{lstlisting}
	\end{frame}
	
	\begin{frame}[fragile]{include vs. input}
		\textbf{Two commands for external files:}
		
		\begin{block}{\textbackslash input\{file\}}
			\begin{itemize}
				\item Simply inserts content
				\item No new page
				\item Use for any content
			\end{itemize}
		\end{block}
		
		\begin{block}{\textbackslash include\{file\}}
			\begin{itemize}
				\item Starts new page before/after
				\item Can use \texttt{\textbackslash includeonly}
				\item Use for chapters
			\end{itemize}
		\end{block}
		
		\begin{lstlisting}
			@\textbackslash@includeonly{chapter2,chapter3}
		\end{lstlisting}
	\end{frame}
	
	\begin{frame}{Project Structure}
		\textbf{Recommended organization:}
		
		\begin{itemize}
			\item \texttt{thesis/}
			\begin{itemize}
				\item \texttt{main.tex}
				\item \texttt{preamble.tex}
				\item \texttt{references.bib}
				\item \texttt{chapters/}
				\begin{itemize}
					\item \texttt{chapter1.tex}
					\item \texttt{chapter2.tex}
				\end{itemize}
				\item \texttt{figures/}
				\item \texttt{tables/}
			\end{itemize}
		\end{itemize}
	\end{frame}
	
	%==============================================================================
	\section{Reports and Theses}
	
	\begin{frame}[fragile]{Report Document Class}
		The \texttt{report} class for longer documents:
		
		\begin{lstlisting}
			@\textbackslash@documentclass[12pt,a4paper]{report}
			@\textbackslash@usepackage[margin=1in]{geometry}
			@\textbackslash@usepackage{graphicx}
			@\textbackslash@usepackage{natbib}
			
			@\textbackslash@title{Research Report}
			@\textbackslash@author{Your Name}
			@\textbackslash@date{@\textbackslash@today}
			
			@\textbackslash@begin{document}
			@\textbackslash@maketitle
			@\textbackslash@tableofcontents
			@\textbackslash@chapter{Introduction}
		\end{lstlisting}
	\end{frame}
	
	\begin{frame}[fragile]{Title Page and Front Matter}
		\begin{lstlisting}
			@\textbackslash@begin{document}
			
			% Title page
			@\textbackslash@maketitle
			
			% Abstract
			@\textbackslash@begin{abstract}
			Research summary...
			@\textbackslash@end{abstract}
			
			% Contents
			@\textbackslash@tableofcontents
			@\textbackslash@listoffigures
			@\textbackslash@listoftables
		\end{lstlisting}
	\end{frame}
	
	\begin{frame}[fragile]{Chapter vs. Section}
		\textbf{In report/book classes:}
		
		\begin{lstlisting}
			@\textbackslash@chapter{Introduction}
			@\textbackslash@section{Background}
			@\textbackslash@subsection{Context}
			@\textbackslash@subsubsection{Details}
		\end{lstlisting}
		
		\vspace{1em}
		
		\textbf{Numbering:}
		\begin{itemize}
			\item Chapters: 1, 2, 3, ...
			\item Sections: 1.1, 1.2, 2.1, ...
			\item Subsections: 1.1.1, 1.1.2, ...
		\end{itemize}
	\end{frame}
	
	\begin{frame}[fragile]{Thesis-Specific Elements}
		\begin{lstlisting}
			% Dedication
			@\textbackslash@chapter*{Dedication}
			To my family...
			
			% Acknowledgments
			@\textbackslash@chapter*{Acknowledgments}
			I thank...
			
			% Appendices
			@\textbackslash@appendix
			@\textbackslash@chapter{Survey Data}
			@\textbackslash@chapter{Additional Results}
		\end{lstlisting}
		
		\begin{alertblock}{Note}
			After \texttt{\textbackslash appendix}, chapters labeled A, B, C...
		\end{alertblock}
	\end{frame}
	
	\begin{frame}{SUZA Thesis Requirements}
		\textbf{Check with your department:}
		\begin{itemize}
			\item Official SUZA thesis template
			\item Formatting requirements
			\item Title page layout
			\item Citation style
		\end{itemize}
		
		\vspace{1em}
		
		\textbf{Common Requirements:}
		\begin{itemize}
			\item Double spacing: \texttt{\textbackslash usepackage\{setspace\}}
			\item Specific margins: \texttt{\textbackslash usepackage\{geometry\}}
			\item Page numbering styles
		\end{itemize}
	\end{frame}
	
	%==============================================================================
	\section{Creating Presentations}
	
	\begin{frame}{Introduction to Beamer}
		\begin{block}{What is Beamer?}
			LaTeX document class for presentation slides.
		\end{block}
		
		\vspace{1em}
		
		\textbf{Advantages:}
		\begin{itemize}
			\item Professional, consistent design
			\item Easy math formulas
			\item Version control friendly
			\item Reuse content from papers
			\item Automatic navigation
		\end{itemize}
		
		\vspace{1em}
		
		\textbf{This presentation is made with Beamer!}
	\end{frame}
	
	\begin{frame}[fragile]{Basic Beamer Structure}
		\begin{lstlisting}
			@\textbackslash@documentclass{beamer}
			@\textbackslash@usetheme{Madrid}
			
			@\textbackslash@title{My Presentation}
			@\textbackslash@author{Your Name}
			@\textbackslash@date{@\textbackslash@today}
			
			@\textbackslash@begin{document}
			
			@\textbackslash@frame{@\textbackslash@titlepage}
			
			@\textbackslash@begin{frame}{First Slide}
			Content here...
			@\textbackslash@end{frame}
			
			@\textbackslash@end{document}
		\end{lstlisting}
	\end{frame}
	
	\begin{frame}{Beamer Themes}
		\textbf{Choose with} \texttt{\textbackslash usetheme\{\}}:
		
		\begin{columns}
			\begin{column}{0.48\textwidth}
				\textbf{Popular Themes:}
				\begin{itemize}
					\item Madrid
					\item Berlin
					\item Copenhagen
					\item Warsaw
					\item Singapore
				\end{itemize}
			\end{column}
			
			\begin{column}{0.48\textwidth}
				\textbf{Color Themes:}
				\begin{itemize}
					\item default
					\item beaver
					\item crane
					\item dolphin
					\item orchid
				\end{itemize}
			\end{column}
		\end{columns}
	\end{frame}
	
	\begin{frame}[fragile]{Slide Elements}
		\begin{lstlisting}
			@\textbackslash@begin{frame}{Title}
			
			@\textbackslash@begin{block}{Block Title}
			Important content in box.
			@\textbackslash@end{block}
			
			@\textbackslash@begin{alertblock}{Warning}
			Pay attention!
			@\textbackslash@end{alertblock}
			
			@\textbackslash@begin{exampleblock}{Example}
			Example here...
			@\textbackslash@end{exampleblock}
			
			@\textbackslash@end{frame}
		\end{lstlisting}
	\end{frame}
	
	\begin{frame}[fragile]{Columns and Lists}
		\begin{lstlisting}
			@\textbackslash@begin{frame}{Two Columns}
			@\textbackslash@begin{columns}
			@\textbackslash@begin{column}{0.48@\textbackslash@textwidth}
			Left content
			@\textbackslash@end{column}
			@\textbackslash@begin{column}{0.48@\textbackslash@textwidth}
			Right content
			@\textbackslash@end{column}
			@\textbackslash@end{columns}
			@\textbackslash@end{frame}
		\end{lstlisting}
	\end{frame}
	
	\begin{frame}[fragile]{Overlays and Animations}
		\begin{lstlisting}
			@\textbackslash@begin{frame}{Progressive}
			@\textbackslash@begin{itemize}
			@\textbackslash@item<1-> First appears
			@\textbackslash@item<2-> Second appears
			@\textbackslash@item<3-> Third appears
			@\textbackslash@end{itemize}
			
			@\textbackslash@pause
			After pause.
			
			@\textbackslash@only<1>{Visible slide 1}
			@\textbackslash@only<2>{Visible slide 2}
			@\textbackslash@end{frame}
		\end{lstlisting}
	\end{frame}
	
	\begin{frame}[fragile]{Table of Contents}
		\begin{lstlisting}
			@\textbackslash@begin{frame}{Outline}
			@\textbackslash@tableofcontents
			@\textbackslash@end{frame}
			
			% Highlight current section
			@\textbackslash@begin{frame}{Outline}
			@\textbackslash@tableofcontents[currentsection]
			@\textbackslash@end{frame}
			
			% Define sections
			@\textbackslash@section{Introduction}
			@\textbackslash@section{Methods}
		\end{lstlisting}
	\end{frame}
	
	%==============================================================================
	\section{Custom Commands}
	
	\begin{frame}{Why Custom Commands?}
		\textbf{Benefits:}
		\begin{itemize}
			\item Reduce repetitive typing
			\item Ensure consistency
			\item Easy global updates
			\item More readable code
		\end{itemize}
		
		\vspace{1em}
		
		\textbf{Example:} Instead of typing\\
		\texttt{\textbackslash textbf\{\textbackslash textit\{\textbackslash textcolor\{blue\}\{text\}\}\}}\\
		every time, create:\\
		\texttt{\textbackslash highlight\{text\}}
	\end{frame}
	
	\begin{frame}[fragile]{Defining Commands}
		\begin{lstlisting}
			% Simple (no arguments)
			@\textbackslash@newcommand{@\textbackslash@suza}{State University of Zanzibar}
			
			% One argument
			@\textbackslash@newcommand{@\textbackslash@dept}[1]{Department of #1}
			
			% Multiple arguments
			@\textbackslash@newcommand{@\textbackslash@super}[2]{Supervisor: #1, #2}
			
			% Using them
			@\textbackslash@suza{}
			@\textbackslash@dept{Computer Science}
			@\textbackslash@super{Dr. Smith}{SUZA}
		\end{lstlisting}
	\end{frame}
	
	\begin{frame}[fragile]{Math Commands}
		\begin{lstlisting}
			% Expectation
			@\textbackslash@newcommand{@\textbackslash@E}{@\textbackslash@mathbb{E}}
			
			% Probability
			@\textbackslash@newcommand{@\textbackslash@Prob}{@\textbackslash@mathbb{P}}
			
			% Vectors
			@\textbackslash@newcommand{@\textbackslash@vect}[1]{@\textbackslash@mathbf{#1}}
			
			% Usage
			$@\textbackslash@E[X]$ is expected value.
			$@\textbackslash@Prob(A)$ is probability.
			$@\textbackslash@vect{x} = (x_1, x_2)$
		\end{lstlisting}
	\end{frame}
	
	\begin{frame}[fragile]{Renewing Commands}
		\begin{lstlisting}
			% Modify existing commands
			@\textbackslash@renewcommand{@\textbackslash@vec}[1]{@\textbackslash@mathbf{#1}}
			
			% Custom section formatting
			@\textbackslash@renewcommand{@\textbackslash@section}[1]{
				@\textbackslash@vspace{1em}
				@\textbackslash@textbf{@\textbackslash@Large #1}
				@\textbackslash@vspace{0.5em}
			}
		\end{lstlisting}
		
		\begin{alertblock}{Warning}
			Be careful when renewing standard commands!
		\end{alertblock}
	\end{frame}
	
	%==============================================================================
	\section{Advanced Formatting}
	
	\begin{frame}[fragile]{Professional Tables}
		Use \texttt{booktabs} package:
		
		\begin{lstlisting}
			@\textbackslash@usepackage{booktabs}
			
			@\textbackslash@begin{table}
			@\textbackslash@centering
			@\textbackslash@begin{tabular}{lcc}
			@\textbackslash@toprule
			Method & Accuracy & F1 \\
			@\textbackslash@midrule
			Baseline & 0.85 & 0.82 \\
			Proposed & 0.92 & 0.90 \\
			@\textbackslash@bottomrule
			@\textbackslash@end{tabular}
			@\textbackslash@end{table}
		\end{lstlisting}
	\end{frame}
	
	\begin{frame}[fragile]{Subfigures}
		\begin{lstlisting}
			@\textbackslash@usepackage{subcaption}
			
			@\textbackslash@begin{figure}
			@\textbackslash@begin{subfigure}{0.45@\textbackslash@textwidth}
			@\textbackslash@includegraphics[width=@\textbackslash@textwidth]{plot1.pdf}
			@\textbackslash@caption{First}
			@\textbackslash@end{subfigure}
			@\textbackslash@begin{subfigure}{0.45@\textbackslash@textwidth}
			@\textbackslash@includegraphics[width=@\textbackslash@textwidth]{plot2.pdf}
			@\textbackslash@caption{Second}
			@\textbackslash@end{subfigure}
			@\textbackslash@caption{Comparison}
			@\textbackslash@end{figure}
		\end{lstlisting}
	\end{frame}
	
	\begin{frame}[fragile]{Line Spacing and Margins}
		\begin{lstlisting}
			% Line spacing
			@\textbackslash@usepackage{setspace}
			@\textbackslash@doublespacing  % or @\textbackslash@onehalfspacing
			
			% Custom margins
			@\textbackslash@usepackage[top=1in, bottom=1in,
			left=1.25in, right=1in]{geometry}
			
			% Paragraph spacing
			@\textbackslash@setlength{@\textbackslash@parskip}{1em}
		\end{lstlisting}
	\end{frame}
	
	\begin{frame}[fragile]{Headers and Footers}
		\begin{lstlisting}
			@\textbackslash@usepackage{fancyhdr}
			@\textbackslash@pagestyle{fancy}
			
			@\textbackslash@fancyhead[L]{Left Header}
			@\textbackslash@fancyhead[C]{Center}
			@\textbackslash@fancyhead[R]{Right}
			
			@\textbackslash@fancyfoot[C]{@\textbackslash@thepage}
			
			% Remove lines
			@\textbackslash@renewcommand{@\textbackslash@headrulewidth}{0pt}
		\end{lstlisting}
	\end{frame}
	
	%==============================================================================
	\section{Collaborative Writing}
	
	\begin{frame}{Overleaf Collaboration}
		\textbf{Sharing Projects:}
		\begin{itemize}
			\item Click "Share" in project
			\item Invite by email
			\item Set permissions (view/edit/owner)
			\item Real-time collaboration
		\end{itemize}
		
		\vspace{1em}
		
		\textbf{Version History:}
		\begin{itemize}
			\item Automatic tracking
			\item View previous versions
			\item Compare side-by-side
			\item Add labels
		\end{itemize}
	\end{frame}
	
	\begin{frame}[fragile]{Comments and Tracking}
		\begin{lstlisting}
			% Comments in code
			% TODO: Add details
			% FIXME: Check citation
			
			% Or use todonotes
			@\textbackslash@usepackage{todonotes}
			
			@\textbackslash@todo{Review this}
			@\textbackslash@todo[inline]{Add figure}
			
			% Track changes
			@\textbackslash@usepackage{changes}
			@\textbackslash@added{New text}
			@\textbackslash@deleted{Old text}
		\end{lstlisting}
	\end{frame}
	
	\begin{frame}{Git for LaTeX}
		\textbf{Why Git?}
		\begin{itemize}
			\item Complete version control
			\item Offline work
			\item Branching for revisions
			\item Backup and collaboration
		\end{itemize}
		
		\vspace{1em}
		
		\textbf{Best Practices:}
		\begin{itemize}
			\item Commit frequently
			\item Use \texttt{.gitignore} for aux files
			\item One sentence per line
			\item Tag releases
		\end{itemize}
	\end{frame}
	
	%==============================================================================
	\section{Hands-On Practice}
	
	\begin{frame}{Exercise 1: Bibliography}
		\textbf{Task:}
		\begin{enumerate}
			\item Create \texttt{references.bib} with 5 entries
			\item Create document citing all
			\item Use \texttt{natbib}
			\item Try different styles
			\item Compile successfully
		\end{enumerate}
		
		\vspace{1em}
		\textbf{Time: 20 minutes}
		
		\vspace{0.5em}
		\textit{Hint: Use Google Scholar's "Cite" for BibTeX!}
	\end{frame}
	
	\begin{frame}{Exercise 2: Report Template}
		\textbf{Task:}
		\begin{enumerate}
			\item Multi-chapter report with:
			\begin{itemize}
				\item Title page with abstract
				\item Table of contents
				\item 3 chapters
				\item Table and figure
				\item Bibliography
			\end{itemize}
			\item Use \texttt{report} class
			\item Split chapters into files
			\item Add custom command
		\end{enumerate}
		
		\vspace{1em}
		\textbf{Time: 25 minutes}
	\end{frame}
	
	\begin{frame}{Exercise 3: Beamer}
		\textbf{Task:}
		\begin{enumerate}
			\item 5-slide presentation about research
			\item Include:
			\begin{itemize}
				\item Title slide
				\item Table of contents
				\item Blocks and columns
				\item Equation
				\item Progressive reveal
			\end{itemize}
			\item Choose and customize theme
		\end{enumerate}
		
		\vspace{1em}
		\textbf{Time: 20 minutes}
	\end{frame}
	
	%==============================================================================
	\section{Tips and Best Practices}
	
	\begin{frame}{Writing Efficiently}
		\textbf{Organization:}
		\begin{itemize}
			\item Consistent naming
			\item One sentence per line
			\item Comment generously
			\item Meaningful labels
		\end{itemize}
		
		\vspace{1em}
		
		\textbf{Performance:}
		\begin{itemize}
			\item Use \texttt{\textbackslash includeonly}
			\item Compress images
			\item Draft mode
		\end{itemize}
	\end{frame}
	
	\begin{frame}{Common Pitfalls}
		\begin{enumerate}
			\item Hardcoding numbers
			\item Manual spacing
			\item Inconsistent formatting
			\item Ignoring warnings
			\item Not backing up
			\item Wrong figure formats
		\end{enumerate}
	\end{frame}
	
	\begin{frame}{Resources}
		\textbf{Documentation:}
		\begin{itemize}
			\item CTAN: \url{https://www.ctan.org}
			\item Package docs: \texttt{texdoc <package>}
		\end{itemize}
		
		\vspace{0.5em}
		
		\textbf{Specialized:}
		\begin{itemize}
			\item TikZ: \url{https://tikz.dev}
			\item PGFPlots: \url{https://pgfplots.sourceforge.net}
		\end{itemize}
		
		\vspace{0.5em}
		
		\textbf{Templates:}
		\begin{itemize}
			\item Overleaf: \url{https://www.overleaf.com/latex/templates}
		\end{itemize}
	\end{frame}
	
	%==============================================================================
	\section{Preparing for Day 3}
	
	\begin{frame}{Tomorrow: AI Integration}
		\textbf{Day 3 Topics:}
		\begin{itemize}
			\item AI tools for LaTeX assistance
			\item Generating LaTeX from natural language
			\item AI citation management
			\item Automated generation
			\item Proofreading with AI
			\item Ethics in AI writing
			\item Complete research document
		\end{itemize}
		
		\vspace{1em}
		
		\begin{exampleblock}{Prepare}
			Bring research topic for final project!
		\end{exampleblock}
	\end{frame}
	
	\begin{frame}{Homework (Optional)}
		\textbf{Conference paper template:}
		\begin{itemize}
			\item Two-column \texttt{article}
			\item Title, authors, abstract
			\item Several sections
			\item 2 figures, 1 table
			\item 5 citations
			\item IEEE or ACM style
		\end{itemize}
		
		\vspace{1em}
		
		\textbf{Bonus:}
		Find Overleaf template for your field!
	\end{frame}
	
	%==============================================================================
	\begin{frame}
		\begin{center}
			\Huge\textbf{Thank You!}
			
			\vspace{2em}
			
			\Large Day 2 Complete!
			
			\vspace{1em}
			
			\normalsize
			Questions? Discussion?
			
			\vspace{1em}
			
			massoud.hamad@suza.ac.tz
		\end{center}
	\end{frame}
	
\end{document}