% Day 1: Introduction to LaTeX Basics - COMPLETE FIXED VERSION
% SUZA Workshop - Scientific Writing with LaTeX & AI
% ALL code blocks cleaned, all errors fixed

\documentclass[aspectratio=169]{beamer}

% Theme and Color Settings
\usetheme{Madrid}
\usecolortheme{default}
\setbeamercolor{structure}{fg=blue!70!black}
\setbeamercolor{frametitle}{bg=blue!10,fg=blue!70!black}
\usefonttheme{serif}

% Packages
\usepackage[utf8]{inputenc}
\usepackage{graphicx}
\usepackage{listings}
\usepackage{xcolor}
\usepackage{hyperref}

% Code Listing Settings - FIXED
\lstset{
	basicstyle=\ttfamily\small,
	keywordstyle=\color{blue},
	commentstyle=\color{gray},
	stringstyle=\color{red},
	frame=single,
	breaklines=true,
	numbers=left,
	numberstyle=\tiny\color{gray},
	escapechar=@,
	xleftmargin=0pt,
	framexleftmargin=0pt
}

\usepackage{listings}
\lstset{escapechar=@, basicstyle=\ttfamily\small}

% Header and Footer Configuration
\setbeamertemplate{headline}{
	\leavevmode%
	\hbox{%
		\begin{beamercolorbox}[wd=.02\paperwidth,ht=2.5ex,dp=1.125ex]{section in head/foot}%
		\end{beamercolorbox}%
		\begin{beamercolorbox}[wd=.15\paperwidth,ht=2.5ex,dp=1.125ex,left]{section in head/foot}%
			\includegraphics[height=2ex]{../resources/suza_logo.png}
		\end{beamercolorbox}%
		\begin{beamercolorbox}[wd=.83\paperwidth,ht=2.5ex,dp=1.125ex,center]{section in head/foot}%
			\scriptsize State University of Zanzibar (SUZA) | Scientific Writing with LaTeX \& AI
		\end{beamercolorbox}%
	}
	\vskip2pt
	{\color{blue!70!black}\hrule height 0.5pt}
}

\setbeamertemplate{footline}{
	{\color{blue!70!black}\hrule height 0.5pt}
	\vskip2pt
	\leavevmode%
	\hbox{%
		\begin{beamercolorbox}[wd=.5\paperwidth,ht=2.5ex,dp=1.125ex,left,leftskip=1em]{author in head/foot}%
			\scriptsize Masoud Hamad  |  massoud.hamad@suza.ac.tz  |  Department of CS\&IT, SCCMS
		\end{beamercolorbox}%
		\begin{beamercolorbox}[wd=.4\paperwidth,ht=2.5ex,dp=1.125ex,right]{date in head/foot}%
			\scriptsize\insertshortdate
		\end{beamercolorbox}%
		\begin{beamercolorbox}[wd=.1\paperwidth,ht=2.5ex,dp=1.125ex,right,rightskip=1em]{page number in head/foot}%
			\insertframenumber{} / \inserttotalframenumber
		\end{beamercolorbox}
	}%
	\vskip2pt
}

\title{Day 1: Introduction to LaTeX Basics}
\subtitle{Document Structure, Formatting \& Essential Commands}
\author{Masoud Hamad}
\institute{Department of Computer Science \& Information Technology\\
	School of Computing, Communication and Media Studies\\
	State University of Zanzibar}
\date{\today}

\begin{document}
	
	\begin{frame}
		\titlepage
	\end{frame}
	
	\begin{frame}{Today's Agenda}
		\tableofcontents
	\end{frame}
	
	%==============================================================================
	\section{What is LaTeX?}
	
	\begin{frame}{What is LaTeX?}
		\begin{block}{Definition}
			LaTeX is a document preparation system for high-quality typesetting, widely used for scientific and technical documents.
		\end{block}
		
		\vspace{1em}
		
		\textbf{Why Use LaTeX?}
		\begin{itemize}
			\item Professional typesetting, especially for mathematical formulas
			\item Consistent formatting throughout documents
			\item Easy management of references and bibliographies
			\item Industry standard for academic publishing
			\item Free and cross-platform
		\end{itemize}
	\end{frame}
	
	\begin{frame}{LaTeX vs. Word Processors}
		\begin{columns}
			\begin{column}{0.48\textwidth}
				\textbf{Traditional Word Processors}
				\begin{itemize}
					\item WYSIWYG interface
					\item Click-and-drag formatting
					\item Good for simple documents
					\item Formatting can be inconsistent
				\end{itemize}
			\end{column}
			
			\begin{column}{0.48\textwidth}
				\textbf{LaTeX}
				\begin{itemize}
					\item Code-based formatting
					\item Excellent for complex documents
					\item Consistent professional output
					\item Steep learning curve
				\end{itemize}
			\end{column}
		\end{columns}
	\end{frame}
	
	%==============================================================================
	\section{Getting Started}
	
	\begin{frame}{Installing LaTeX}
		\textbf{LaTeX Distributions:}
		\begin{itemize}
			\item \textbf{Windows:} MiKTeX or TeX Live
			\item \textbf{macOS:} MacTeX
			\item \textbf{Linux:} TeX Live (via package manager)
		\end{itemize}
		
		\vspace{1em}
		
		\textbf{Online Options (No Installation):}
		\begin{itemize}
			\item \textbf{Overleaf:} \url{https://www.overleaf.com}
			\item \textbf{Papeeria:} \url{https://papeeria.com}
		\end{itemize}
		
		\vspace{1em}
		
		\begin{alertblock}{Recommendation}
			For today's workshop, we'll use Overleaf for quick access!
		\end{alertblock}
	\end{frame}
	
	\begin{frame}[fragile]{Your First LaTeX Document}
		\textbf{Basic Structure:}
		
		\begin{lstlisting}
			@\textbackslash@documentclass{article}
			
			@\textbackslash@begin{document}
			Hello, World!
			@\textbackslash@end{document}
		\end{lstlisting}
		
		\vspace{1em}
		
		Three essential components:
		\begin{enumerate}
			\item \texttt{\textbackslash documentclass\{article\}} - Document type
			\item \texttt{\textbackslash begin\{document\}} - Content starts
			\item \texttt{\textbackslash end\{document\}} - Content ends
		\end{enumerate}
	\end{frame}
	
	%==============================================================================
	\section{Document Structure}
	
	\begin{frame}[fragile]{Document Classes}
		Choose the right document class:
		
		\begin{lstlisting}
			@\textbackslash@documentclass{article}   % Papers
			@\textbackslash@documentclass{report}    % Reports
			@\textbackslash@documentclass{book}      % Books
			@\textbackslash@documentclass{beamer}    % Slides
		\end{lstlisting}
		
		\vspace{1em}
		
		\textbf{Common Options:}
		\begin{lstlisting}
			@\textbackslash@documentclass[12pt, a4paper]{article}
		\end{lstlisting}
	\end{frame}
	
	\begin{frame}[fragile]{Preamble: Loading Packages}
		The preamble comes before \texttt{\textbackslash begin\{document\}}:
		
		\begin{lstlisting}
			@\textbackslash@documentclass{article}
			
			% Essential packages
			@\textbackslash@usepackage[utf8]{inputenc}
			@\textbackslash@usepackage{graphicx}
			@\textbackslash@usepackage{amsmath}
			
			@\textbackslash@title{My Document}
			@\textbackslash@author{Your Name}
			
			@\textbackslash@begin{document}
			% Content here
			@\textbackslash@end{document}
		\end{lstlisting}
	\end{frame}
	
	\begin{frame}[fragile]{Title, Author, and Date}
		\begin{lstlisting}
			@\textbackslash@documentclass{article}
			
			@\textbackslash@title{Workshop Title}
			@\textbackslash@author{Your Name}
			@\textbackslash@date{@\textbackslash@today}
			
			@\textbackslash@begin{document}
			@\textbackslash@maketitle
			@\textbackslash@end{document}
		\end{lstlisting}
	\end{frame}
	
	%==============================================================================
	\section{Text Formatting}
	
\begin{frame}[fragile]{Basic Text Formatting}
	\begin{columns}
		\begin{column}{0.48\textwidth}
			\textbf{Code:}
			\begin{lstlisting}
				@\textbackslash@textbf{Bold}
				@\textbackslash@textit{Italic}
				@\textbackslash@underline{Under}
				@\textbackslash@texttt{Mono}
			\end{lstlisting}
		\end{column}
		\begin{column}{0.2\textwidth}
			\textbf{Output:}
			\begin{itemize}
				\item \textbf{Bold text}
				\item \textit{Italic text}
				\item \underline{Underlined}
				\item \texttt{Monospace}
			\end{itemize}
		\end{column}
	\end{columns}
\end{frame}
	
	\begin{frame}[fragile]{Sections and Subsections}
		\begin{lstlisting}
			@\textbackslash@section{Introduction}
			This is the introduction.
			
			@\textbackslash@subsection{Background}
			Some background info.
			
			@\textbackslash@subsubsection{Details}
			More detail here.
			
			@\textbackslash@section{Methods}
			Methodology here...
		\end{lstlisting}
		
		\vspace{0.5em}
		
		LaTeX automatically numbers and formats sections.
	\end{frame}
	
	\begin{frame}[fragile]{Lists: Itemize and Enumerate}
		\begin{columns}[t]
			\begin{column}{0.48\textwidth}
				\textbf{Bulleted List:}
				\begin{lstlisting}
					@\textbackslash@begin{itemize}
					@\textbackslash@item First
					@\textbackslash@item Second
					@\textbackslash@item Third
					@\textbackslash@end{itemize}
				\end{lstlisting}
			\end{column}
			
			\begin{column}{0.48\textwidth}
				\textbf{Numbered List:}
				\begin{lstlisting}
					@\textbackslash@begin{enumerate}
					@\textbackslash@item First
					@\textbackslash@item Second
					@\textbackslash@item Third
					@\textbackslash@end{enumerate}
				\end{lstlisting}
			\end{column}
		\end{columns}
	\end{frame}
	
	\begin{frame}[fragile]{Description Lists}
		\begin{lstlisting}
			@\textbackslash@begin{description}
			@\textbackslash@item[Term 1] Definition of term 1
			@\textbackslash@item[Term 2] Definition of term 2
			@\textbackslash@item[Term 3] Definition of term 3
			@\textbackslash@end{description}
		\end{lstlisting}
		
		\vspace{1em}
		
		\textbf{Output:}
		\begin{description}
			\item[LaTeX] Document preparation system
			\item[BibTeX] Bibliography management tool
			\item[Beamer] Presentation package
		\end{description}
	\end{frame}
	
	%==============================================================================
	\section{Mathematical Formulas}
	
	\begin{frame}[fragile]{Inline vs. Display Math}
		\begin{block}{Inline Math}
			Use \texttt{\$...\$} for math within text:
			\begin{lstlisting}
				The equation $E = mc^2$ is famous.
			\end{lstlisting}
			Output: The equation $E = mc^2$ is famous.
		\end{block}
		
		\vspace{1em}
		
		\begin{block}{Display Math}
			\begin{lstlisting}
				@\textbackslash@begin{equation}
				E = mc^2
				@\textbackslash@end{equation}
			\end{lstlisting}
			Output:
			\begin{equation}
				E = mc^2
			\end{equation}
		\end{block}
	\end{frame}
	
	\begin{frame}[fragile]{Common Math Symbols}
		\begin{columns}[t]
			\begin{column}{0.48\textwidth}
				\textbf{Greek Letters:}
				\begin{lstlisting}
					$@\textbackslash@alpha, @\textbackslash@beta, @\textbackslash@gamma$
					$@\textbackslash@Delta, @\textbackslash@theta, @\textbackslash@pi$
				\end{lstlisting}
				Output: $\alpha, \beta, \gamma, \Delta, \theta, \pi$
			\end{column}
			
			\begin{column}{0.48\textwidth}
				\textbf{Operators:}
				\begin{lstlisting}
					$@\textbackslash@sum, @\textbackslash@int, @\textbackslash@prod$
					$@\textbackslash@lim, @\textbackslash@sqrt{x}$
				\end{lstlisting}
				Output: $\sum, \int, \prod, \lim, \sqrt{x}$
			\end{column}
		\end{columns}
	\end{frame}
	
	\begin{frame}[fragile]{Fractions and Subscripts}
		\begin{lstlisting}
			$@\textbackslash@frac{numerator}{denominator}$
			
			$x^2$  % Superscript
			
			$x_i$  % Subscript
			
			$x_i^2$  % Both
		\end{lstlisting}
		
		\vspace{1em}
		
		\textbf{Examples:}
		\begin{itemize}
			\item Fraction: $\frac{a+b}{c+d}$
			\item Power: $x^2 + y^2 = r^2$
			\item Subscript: $x_1, x_2, \ldots, x_n$
		\end{itemize}
	\end{frame}
	
	\begin{frame}[fragile]{Summation and Integration}
		\begin{lstlisting}
			$@\textbackslash@sum_{i=1}^{n} x_i$
			
			$@\textbackslash@int_0^1 f(x) dx$
			
			$@\textbackslash@prod_{k=1}^{n} a_k$
		\end{lstlisting}
		
		\vspace{1em}
		
		\textbf{Display versions:}
		\begin{itemize}
			\item $\displaystyle\sum_{i=1}^{n} x_i$
			\item $\displaystyle\int_0^1 f(x) dx$
			\item $\displaystyle\prod_{k=1}^{n} a_k$
		\end{itemize}
	\end{frame}
	
	\begin{frame}[fragile]{Complex Equations}
		\begin{lstlisting}
			@\textbackslash@begin{equation}
			f(x) = @\textbackslash@int_{-@\textbackslash@infty}^{@\textbackslash@infty}
			@\textbackslash@hat{f}(@\textbackslash@xi) e^{2@\textbackslash@pi i @\textbackslash@xi x} d@\textbackslash@xi
			@\textbackslash@end{equation}
		\end{lstlisting}
		
		\vspace{0.5em}
		
		\textbf{Output:}
		\begin{equation}
			f(x) = \int_{-\infty}^{\infty} \hat{f}(\xi) e^{2\pi i \xi x} d\xi
		\end{equation}
	\end{frame}
	
	\begin{frame}[fragile]{Multi-line Equations}
		\begin{lstlisting}
			@\textbackslash@begin{align}
			x &= a + b \\
			y &= c + d \\
			z &= e + f
			@\textbackslash@end{align}
		\end{lstlisting}
		
		\textbf{Output:}
		\begin{align}
			x &= a + b \\
			y &= c + d \\
			z &= e + f
		\end{align}
	\end{frame}
	
	%==============================================================================
	\section{Tables and Figures}
	
	\begin{frame}[fragile]{Creating Tables}
		\begin{lstlisting}
			@\textbackslash@begin{table}[h]
			@\textbackslash@centering
			@\textbackslash@caption{Sample Data}
			@\textbackslash@begin{tabular}{|c|c|c|}
			@\textbackslash@hline
			Name & Age & Grade \\
			@\textbackslash@hline
			Alice & 22 & A \\
			Bob & 23 & B \\
			@\textbackslash@hline
			@\textbackslash@end{tabular}
			@\textbackslash@end{table}
		\end{lstlisting}
	\end{frame}
	
	\begin{frame}[fragile]{Table Alignment}
		\begin{lstlisting}
			@\textbackslash@begin{tabular}{lcc}
			% l = left, c = center, r = right
		\end{lstlisting}
		
		\vspace{1em}
		
		\begin{table}
			\centering
			\begin{tabular}{lcc}
				\hline
				\textbf{Name} & \textbf{Age} & \textbf{Grade} \\
				\hline
				Alice & 22 & A \\
				Bob & 23 & B+ \\
				Charlie & 21 & A- \\
				\hline
			\end{tabular}
		\end{table}
	\end{frame}
	
	\begin{frame}[fragile]{Including Figures}
		\begin{lstlisting}
			@\textbackslash@begin{figure}[h]
			@\textbackslash@centering
			@\textbackslash@includegraphics[width=0.5@\textbackslash@textwidth]{image.png}
			@\textbackslash@caption{Figure caption}
			@\textbackslash@label{fig:sample}
			@\textbackslash@end{figure}
		\end{lstlisting}
		
		\vspace{1em}
		
		\textbf{Size Options:}
		\begin{itemize}
			\item \texttt{width=0.5\textbackslash textwidth}
			\item \texttt{height=5cm}
			\item \texttt{scale=0.8}
		\end{itemize}
	\end{frame}
	
	\begin{frame}[fragile]{Figure Placement}
		\begin{lstlisting}
			@\textbackslash@begin{figure}[htbp]
			% h = here
			% t = top of page
			% b = bottom of page
			% p = separate page
			% ! = override LaTeX rules
			@\textbackslash@end{figure}
		\end{lstlisting}
		
		\vspace{1em}
		
		\begin{alertblock}{Tip}
			Use \texttt{[!h]} to force placement approximately here.
		\end{alertblock}
	\end{frame}
	
	%==============================================================================
	\section{Cross-References}
	
	\begin{frame}[fragile]{Labels and References}
		\begin{lstlisting}
			@\textbackslash@section{Introduction}
			@\textbackslash@label{sec:intro}
			
			See Section~@\textbackslash@ref{sec:intro} for details.
			
			@\textbackslash@begin{equation}
			E = mc^2
			@\textbackslash@label{eq:einstein}
			@\textbackslash@end{equation}
			
			Equation~@\textbackslash@ref{eq:einstein} shows...
		\end{lstlisting}
		
		\vspace{0.5em}
		
		\textbf{Note:} Use \texttt{\textasciitilde} for non-breaking space before ref.
	\end{frame}
	
	\begin{frame}[fragile]{Label Naming Convention}
		\textbf{Good Practice:}
		
		\begin{lstlisting}
			@\textbackslash@label{sec:introduction}  % Sections
			@\textbackslash@label{fig:results}       % Figures
			@\textbackslash@label{tab:data}          % Tables
			@\textbackslash@label{eq:quadratic}      % Equations
		\end{lstlisting}
		
		\vspace{1em}
		
		\textbf{Benefits:}
		\begin{itemize}
			\item Easy to remember
			\item Avoid naming conflicts
			\item Clear documentation
			\item Better organization
		\end{itemize}
	\end{frame}
	
	%==============================================================================
	\section{Hands-On Practice}
	
	\begin{frame}{Practice Exercise 1}
		\textbf{Create your first document:}
		
		\begin{enumerate}
			\item Open Overleaf, create new project
			\item Add title, author, date
			\item Create 3 sections
			\item Add one bulleted list
			\item Include one equation
			\item Compile and check PDF
		\end{enumerate}
		
		\vspace{1em}
		
		\textbf{Time: 15 minutes}
		
		\vspace{0.5em}
		
		\begin{alertblock}{Tip}
			Start simple! You can always add more later.
		\end{alertblock}
	\end{frame}
	
	\begin{frame}{Practice Exercise 2}
		\textbf{Advanced formatting:}
		
		\begin{enumerate}
			\item Create a 3x4 table with headers
			\item Add a figure (use placeholder or example-image)
			\item Label your sections, table, and figure
			\item Add cross-references in text
			\item Compile successfully
		\end{enumerate}
		
		\vspace{1em}
		
		\textbf{Time: 15 minutes}
		
		\vspace{0.5em}
		
		\begin{block}{Help Available}
			Raise your hand if you need assistance!
		\end{block}
	\end{frame}
	
	%==============================================================================
	\section{Troubleshooting}
	
	\begin{frame}{Common Errors}
		\begin{enumerate}
			\item \textbf{Missing \$ inserted}
			\begin{itemize}
				\item Forgot to close math mode with \$
				\item Special characters need escaping
			\end{itemize}
			
			\item \textbf{Undefined control sequence}
			\begin{itemize}
				\item Misspelled command
				\item Missing package in preamble
			\end{itemize}
			
			\item \textbf{File not found}
			\begin{itemize}
				\item Check image path and filename
				\item Ensure file is uploaded to project
			\end{itemize}
		\end{enumerate}
	\end{frame}
	
	\begin{frame}{Debugging Tips}
		\textbf{When compilation fails:}
		
		\begin{enumerate}
			\item Read error message carefully
			\item Check line number indicated
			\item Look for:
			\begin{itemize}
				\item Unclosed braces \{\}
				\item Missing \texttt{\textbackslash end\{...\}}
				\item Special characters without backslash
			\end{itemize}
			\item Comment out recent changes
			\item Search error online
		\end{enumerate}
		
		\vspace{1em}
		
		\begin{block}{Pro Tip}
			Compile frequently! Easier to find errors when you know what changed.
		\end{block}
	\end{frame}
	
	%==============================================================================
	\section{Resources}
	
	\begin{frame}{Learning Resources}
		\textbf{Documentation \& Tutorials:}
		\begin{itemize}
			\item Overleaf Learn: \url{https://www.overleaf.com/learn}
			\item LaTeX Wikibook: \url{https://en.wikibooks.org/wiki/LaTeX}
			\item CTAN: \url{https://www.ctan.org}
		\end{itemize}
		
		\vspace{0.5em}
		
		\textbf{Q\&A Communities:}
		\begin{itemize}
			\item TeX Stack Exchange: \url{https://tex.stackexchange.com}
			\item r/LaTeX: \url{https://reddit.com/r/LaTeX}
		\end{itemize}
		
		\vspace{0.5em}
		
		\textbf{Tools:}
		\begin{itemize}
			\item Detexify (draw symbols): \url{http://detexify.kirelabs.org}
			\item Tables Generator: \url{https://www.tablesgenerator.com}
		\end{itemize}
	\end{frame}
	
	\begin{frame}{Day 2 Preview}
		\textbf{Tomorrow we'll cover:}
		
		\begin{itemize}
			\item Bibliography management with BibTeX
			\item Advanced document structuring
			\item Custom commands and environments
			\item Multi-file projects
			\item Professional reports and theses
			\item Beamer presentations
		\end{itemize}
		
		\vspace{1em}
		
		\begin{block}{Optional Homework}
			Create a 2-page document about your research topic with sections, equations, a table, and a figure.
		\end{block}
	\end{frame}
	
	%==============================================================================
	\begin{frame}
		\begin{center}
			\Huge\textbf{Thank You!}
			
			\vspace{2em}
			
			\Large Questions?
			
			\vspace{1em}
			
			\normalsize
			massoud.hamad@suza.ac.tz
			
			\vspace{2em}
			
			\large
			See you tomorrow for Day 2!
		\end{center}
	\end{frame}
	
\end{document}